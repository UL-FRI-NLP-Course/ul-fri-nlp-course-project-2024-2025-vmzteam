%%%%%%%%%%%%%%%%%%%%%%%%%%%%%%%%%%%%%%%%%
% FRI Data Science_report LaTeX Template
% Version 1.0 (28/1/2020)
% 
% Jure Demšar (jure.demsar@fri.uni-lj.si)
%
% Based on MicromouseSymp article template by:
% Mathias Legrand (legrand.mathias@gmail.com) 
% With extensive modifications by:
% Antonio Valente (antonio.luis.valente@gmail.com)
%
% License:
% CC BY-NC-SA 3.0 (http://creativecommons.org/licenses/by-nc-sa/3.0/)
%
%%%%%%%%%%%%%%%%%%%%%%%%%%%%%%%%%%%%%%%%%


%----------------------------------------------------------------------------------------
%	PACKAGES AND OTHER DOCUMENT CONFIGURATIONS
%----------------------------------------------------------------------------------------
\documentclass[fleqn,moreauthors,10pt]{ds_report}
\usepackage[english]{babel}

\graphicspath{{fig/}}




%----------------------------------------------------------------------------------------
%	ARTICLE INFORMATION
%----------------------------------------------------------------------------------------

% Header
\JournalInfo{FRI Natural language processing course 2025}

% Interim or final report
\Archive{Project report} 
%\Archive{Final report} 

% Article title
\PaperTitle{Automatic generation of Slovenian traffic news for RTV Slovenija} 

% Authors (student competitors) and their info
\Authors{Vid Kobal, Matija Tomažič, and Žan Luka Nusdorfer}

% Advisors
\affiliation{\textit{Advisors: Slavko Žitnik}}

% Keywords
\Keywords{prompt engineering, text generation, traffic news, fine-tuning, data-to-text generation, RTV Slovenija}
\newcommand{\keywordname}{Keywords}


%----------------------------------------------------------------------------------------
%	ABSTRACT
%----------------------------------------------------------------------------------------

\Abstract{
This project aims to fine-tune an existing large language model (LLM) for the task of generating concise and relevant traffic news reports. The goal is to automate the production of traffic updates for radio broadcast, based on structured data from the promet.si portal, a task currently being done manually by students at RTV Slovenija. We use various prompt engineering techniques and evaluate the effectiveness of outputs with (semi) automatic and human evaluation metrics. We take the following into account; accurate road naming, filtering of relevant events, correct filtering, text lengths and words and editorial guidelines. The report also discusses the preparation of datasets, challenges of generation in a low-resource language, and the design of an interface for interactive testing of model outputs.
}

%----------------------------------------------------------------------------------------

\begin{document}

% Makes all text pages the same height
\flushbottom 

% Print the title and abstract box
\maketitle 

% Removes page numbering from the first page
\thispagestyle{empty} 

%----------------------------------------------------------------------------------------
%	ARTICLE CONTENTS
%----------------------------------------------------------------------------------------

\section*{Introduction}
In an effort to modernize and automate the generation of traffic news for RTV Slovenija, this project explores the application of natural language processing (NLP) techniques for the automatic generation of Slovenian traffic reports. The main goal is to use a large language model (LLM), fine-tune it for this specific use case and to use prompt engineering for producing short and relevant traffic updates based on structured data from the promet.si portal. These reports are intended for radio presenters who regularly read them to inform the public.\newline

The task is currently being done manually. While functional, this approach is more labor-intensive, prone to variability, and limited in terms of scalability. By automating the process, we hope to improve efficiency, consistency, and content quality, while following the editorial standards for language style, relevance, and format.\newline

One of the things we focus on is the correct identification of relevant traffic events, accurate road naming, and appropriate text length and filtering. We achieve this with prompt engineering and parameter-efficient fine-tuning (e.g., LoRA). We evaluate generated text using automatic (precision, recall, F1) as well as human evaluation metrics. The project also touches on the design of interactive tools for testing model outputs, and explores best practices in dataset preparation and model evaluation.\newline

In summary, this project aims to deliver a reliable and efficient system for automated traffic news generation by leveraging Large Language Models (LLMs), their fine-tuning, and prompt engineering techniques.


\section*{Related work}
Generating informative traffic new from structured data falls within the broader domain of data-to-text (D2T) generation, specifically under table-to-text generation. For our project, structured traffic data is supplied by the promet.si portal in CSV format. The data includes fields such as road names, event types, timestamps and locations. The objective is to transform these data entries into concise and reliable reports that are ready for radio broadcasters to read.

Recent research has demonstrated that LLMs, when combined with data cleaning and model fine-tuning, are highly effective for table-to-text generation. For example, the HeLM (Highlighted Evidence augmented Language Model) framework \cite{bian2024helmhighlightedevidenceaugmented} proposes a two-step approach, where a highlighter selects more important rows. The highlighter is then followed by a summarizer that produces human-readable text. This method achieved state-of-the-art BLUE (bilingual evaluation understudy) and ROGUE (Recall-Oriented Understudy for Gisting Evaluation) scores, confirming that pre-selecting important data improves both model performance and text understandability.

Using LLMs in low resource environments requires parameter-efficient fine-tuning. Methods like LoRA (Low Rank Adaptation) \cite{lowrandadaptation} and QLoRA (Quantized Low Rank Adaptation) \cite{qlora} allow fine-tuning models on modest hardware while maintaining high accuracy. Researchers have also found that smaller, fine-tuned models ensure reliability and resource efficiency \cite{mahapatra2024impactmodelsizefinetuned}.

Another essential part of our project is exploring and testing different prompt engineering techniques. This involves carefully crafting prompts to elicit specific, high-quality outputs from LLMs. Many studies have been done on prompt engineering techniques, in which researches test different prompt engineering techniques (zero-shot and few-shot prompting, chain-of-thought reasoning, and self-consistency decoding) \cite{vatsal2024surveypromptengineeringmethods}. In traffic news generation, these strategies help guide the model, while also helping it avoid unnecessary or irrelevant information.

Practical guidance for prompt engineering is provided by Prompting Guide AI (\url{https://www.promptingguide.ai/}), which offers extensive explanations and specific advice on different prompt engineering approaches. In addition, Ollama has demonstrated effective prompt workflow with models like Code Llama, especially for tasks involving structure data transformation. They offer different examples on how to prompt Code Llama to optimize relevance and clarity (\url{https://ollama.com/blog/how-to-prompt-code-llama}).

In conclusion, our project builds upon this body of research in table-to-text generation, fine-tuning methods, and prompt engineering. By integrating structured data processing, efficient LLM adaptation, and strategic prompt design, we aim to deliver automated and reliable compliant traffic news that meets the standards of RTV Slovenija for relevance, clarity, and factual accuracy.

\section*{Methods}
The simplest way to tackle this challenge is through prompt engineering, which has been shown to significantly enhance the performance of LLMs.
To improve the quality of our traffic reports, we will leverage various prompt engineering techniques such as few-shot prompting, which provides examples to guide the model’s responses; chain-of-thought prompting, which encourages step-by-step reasoning for better accuracy; and prompt chaining, where multiple prompts are linked together to refine and build upon previous outputs.
These techniques help ensure more accurate, detailed, and contextually relevant traffic reports.

\subsection{Zero-shot prompting}
Zero-shot prompting is a technique where we provide the model with a task description and ask it to generate a response without any examples.
This strategy relies on the model being adequately trained and being exposed to large amounts of data, which allows it to generalize and understand the task at hand.

TODO: example?


\subsection*{Few-shot prompting}
Few-shot prompting involves providing the model with a few examples of the desired input-output format before asking it to generate a response.
It helps the model understand the pattern of the expected output.

For example, we want to classify statements as \textbf{Positive}, \textbf{Negative}, or \textbf{Neutral}.
We provide the LLM with the required task and some examples:
\begin{center}
    \texttt{Task: Classify sentences as Positive, Negative, or Neutral.\newline
    Example 1: 'I hate being in Thailand.' - Negative,\newline
    Example 2: 'I like it when my sister visits.' - Positive}
\end{center}
Then we provide a new sentence that we want to classify:
\begin{center}
    \texttt{Now classify the following sentence: 'I enjoy long walks.'}
\end{center}
The LLM is more likely to classify the sentence correctly, since it has been trained by examples.
This method is very useful for classification, translation, and text generation.

\subsection*{Chain-of-thought prompting (CoT)}
Chain-of-thought prompting is a technique that encourages the model to break down its reasoning process step by step.
By asking the model to explain its thought process, we provide an intermediate reasoning step to guide model's responses, potentially leading to better results.

TODO: example?


\subsection{Prompt chaining}
Prompt chaining is a technique where we link multiple prompts together to refine and build upon the model's previous outputs.
This approach allows us to create a more structured and coherent response by guiding the model through a series of related tasks or questions.
In our case we can first ask the model to generate a list of relevant traffic events, and then ask it to summarize those events into a report.
TODO: pregledat tazadnji stavek, če bomo sploh tako delali


\subsection{LLM fine-tuning}
Another way we can tackle this problem is by fine-tuning an existing LLM model.
This approach allows us to adapt the model to our specific task and domain, in our case generating traffic reports, by training it on a smaller dataset of Slovenian traffic reports, helping the model learn the specific language patterns, terminology, and style used in this context.
We will use the LoRA (Low-Rank Adaptation) method, which is a parameter-efficient fine-tuning technique.
It works by injecting a low-rank trainable matrix into each layer of the pre-trained model, allowing us to fine-tune only a small number of parameters while keeping the rest of the model frozen.
Empirical studies have shown that LoRA can achieve comparable performance to full fine-tuning while significantly reducing the number of trainable parameters and computational resources required.

\subsection{Data preparation}
The data we will use for training and evaluation is sourced from the promet.si portal, which is the main source of traffic information in Slovenia, and will be used as input data and also RTV Slovenija's traffic news archive and formatting guidelines, which will server as expected output data and guidelines for prompting.

\section*{Evaluation}
To evaluate the performance of the different approaches, we will use a combination of automatic and human evaluation metrics.
For automated evaluation, we decided to use BLEU (bilingual evaluation understudy), which compares the machine generated text to one or more reference texts and produces a score between 0 and 1, with values closer to 1 representing more similar texts.
Another automatic evaluation metric we will be using is BERTScore, which uses BERT embeddings to measure the similarity between the generated text and reference texts.
Additionally, we will use manual evaluation, where human evaluators will assess the generated traffic reports based on criteria such as relevance, clarity, and adherence to editorial guidelines.
We will use automatic evaluation metrics on all generated reports, and then select a small subset of the reports for human evaluation, to better understand the correlation between the evaluation methods.
\bibliographystyle{unsrt}
\bibliography{report}

\end{document}